\usepackage{algorithm}
\usepackage{algorithmic}
\usepackage{amsmath}
\usepackage{amssymb}
\usepackage{array}
\usepackage{blindtext}
\usepackage{booktabs}
\usepackage{colortbl}
\usepackage{comment}
\usepackage{etex}
\usepackage{float}
%\usepackage{flafter}
\usepackage[T1]{fontenc}
\usepackage{graphicx}
\usepackage[space]{grffile}
\usepackage[utf8]{inputenc}
\usepackage{multirow}
\usepackage{nag}
\usepackage{pgfplots}
\usepackage{pgfplotstable}
\usepackage{standalone}
\usepackage{tikz}
\usepackage{todonotes}
\usepackage{xspace}
\usepackage[bookmarks=false,pagebackref=false]{hyperref} %Hyperref should go last

%float setup
\newfloat{algorithm}{t}{}

%Set graphics path
\graphicspath{{figures//}}

%hyperref setup
\hypersetup{breaklinks=true}
\hypersetup{colorlinks=true}
\hypersetup{draft=false}
\hypersetup{linktoc=all}
\hypersetup{urlcolor=blue}

\urlstyle{rm}

%tikz setup
\usetikzlibrary{arrows,backgrounds,calc,external,fit,shapes}
\usetikzlibrary{pgfplots.groupplots}

\tikzstyle{bbox}=[draw,blue,thick,inner sep=0]
\tikzstyle{arrow}=[-latex]

\tikzstyle{component}=[draw,rounded corners, minimum size=0.5cm]
\tikzstyle{componentedge}=[bend left=40]
\tikzstyle{componentpath}=[-latex]
\tikzstyle{vote}=[draw,thick,-latex]
\tikzstyle{part}=[circle,draw,fill=white,minimum size=0.2cm,inner sep=0]
\tikzstyle{correctMatch}=[part,fill=green]
\tikzstyle{wrongMatch}=[part,fill=red]
\tikzstyle{candidate}=[part,fill=white]
\tikzstyle{uncertainty}=[circle,fill=lightgray,minimum size=1cm,inner sep=0]
\tikzstyle{centerPoint}=[draw,fill=white]
\tikzstyle{keypoint}=[draw, fill=white,circle,inner sep=0,minimum size=0.10cm]

\tikzexternalize[prefix=ext/]
%\tikzexternaldisable

\pgfplotsset{compat=1.7}
\pgfplotsset{filter discard warning=false}
\pgfplotsset{major grid style={dotted}}

%Set graphics path
\graphicspath{{figures//}}

\DeclareMathOperator*{\argmin}{arg\,min}
\DeclareMathOperator*{\argmax}{arg\,max}

%Make require and ensure input and output
\renewcommand{\algorithmicrequire}{\textbf{Input:}}
\renewcommand{\algorithmicensure}{\textbf{Output:}}

%This can be used to add external file dependencies for latexmk
\makeatletter
\newcommand*{\addFileDependency}[1]{% argument=file name and extension
  \typeout{(#1)}% latexmk will find this if $recorder=0 (however, in that case, it will ignore #1 if it is a .aux or .pdf file etc and it exists! if it doesn't exist, it will appear in the list of dependents regardless)
  \@addtofilelist{#1}% if you want it to appear in \listfiles, not really necessary and latexmk doesn't use this
  \IfFileExists{#1}{}{\typeout{No file #1.}}% latexmk will find this message if #1 doesn't exist (yet)
}
\makeatother

\newcommand{\mytodo}[1]{\textcolor{red}{#1}}
%\newcommand{\mytodo}[1]{}
\newcommand{\imagepath}[0]{figures/results/}
\newcommand{\qualOutputX}[4]{
	\begin{tikzpicture}[scale=1, transform shape]%
      \clip(1,0) rectangle (2,2);
  \begin{axis}[clip marker paths=true, enlargelimits=false, ticks=none, axis equal image, width=5.05cm]%
    \addplot graphics[xmin=0,ymin=0,xmax=#2,ymax=#3] {\imagepath/#1/input_#4.png};%
    \addplot [mark=*,mark size=1,only marks, black,fill=white] table [x=x, y expr=#3-\thisrow{y}] {\imagepath/#1/keypoints_#4.csv};%
%    \addplot [mark=*,mark size=1,only marks, fill=red] table [x=x, y expr=#3-\thisrow{y}] {\imagepath/#1/outliers_#4.csv};%
%    \addplot [thick, mark=none, blue] table [x=x, y expr=#3-\thisrow{y}] {\imagepath/#1/bbox_#4.csv};%
  \end{axis}%
\end{tikzpicture}%
}
\newcommand{\qualOutput}[4]{
\begin{tikzpicture}%
  \begin{axis}[clip marker paths=true, enlargelimits=false, ticks=none, axis equal image, width=4.00cm]%
    \addplot graphics[xmin=0,ymin=0,xmax=#2,ymax=#3] {\imagepath/#1/input_#4.png};%
%    \addplot [mark=*,mark size=1,only marks, fill=white] table [x=x, y expr=#3-\thisrow{y}] {\imagepath/#1/keypoints_#4.csv};%
%    \addplot [mark=*,mark size=1,only marks, fill=red] table [x=x, y expr=#3-\thisrow{y}] {\imagepath/#1/outliers_#4.csv};%
    \addplot [thick, mark=none, blue] table [x=x, y expr=#3-\thisrow{y}] {\imagepath/#1/bbox_#4.csv};%
  \end{axis}%
\end{tikzpicture}%
}
\newcommand{\qualOutputBatch}[6]{%
  \qualOutput{#1}{#2}{#3}{#4}%
	%\hspace{0.2cm}
  \qualOutput{#1}{#2}{#3}{#5}%
	%\hspace{0.2cm}
  \qualOutput{#1}{#2}{#3}{#6}%
}

\tikzstyle{successplotOld}=[mark repeat = 3]

%\newcommand\successplotOld[1]{%
%\begin{tikzpicture}%
%\begin{axis}[xlabel=Required recall,
%    width=6cm,
%ylabel=Success rate,
%grid=major,
%enlargelimits=false,
%yticklabel={\pgfmathprintnumber\tick\%},
%font=\scriptsize]
%%legend style={anchor=north west}]%
%%plot data and legend are generated
%\input{#1}
%\end{axis}%
%\end{tikzpicture}%
%}

\newcommand\successplotDetdesc[1]{%
\begin{tikzpicture}%
\begin{axis}[xlabel=Average overlap,
    width=\linewidth,
    cycle multi list={
      solid,dotted\nextlist
      mark=square*,mark=triangle*,mark=otimes*,mark=diamond*\nextlist
      blue,red,orange,green,black\nextlist
    },
ylabel=Success rate,
%grid=major,
%legend columns=-1,%hide for the moment
enlargelimits=false,
yticklabel={\pgfmathprintnumber\tick\%},
font=\scriptsize,
legend columns=2,
legend style={anchor=north east}]%
%plot data and legend are generated
\input{#1}
\end{axis}%
\end{tikzpicture}%
}

\newcommand\successplotConsensus[1]{%
\begin{tikzpicture}%
\begin{axis}[xlabel=Recall,
    width=5.1cm,
%ylabel=Success rate,
grid=major,
enlargelimits=false,
%yticklabels={\pgfmathprintnumber\tick\},
yticklabels={,,},
font=\scriptsize,
	legend columns=-1,
	legend entries={RANSAC, Hough, FPM},
	legend to name=named]
%legend style={anchor=north west}]%
%plot data and legend are generated
\input{#1}
\end{axis}%
\end{tikzpicture}%
}

\newcommand{\drawPart}[2]{
\node[part] (#1) at (#2) {};
}

\newcommand{\drawUncertainPart}[2]{
	\node[uncertainty] at (#2) {};
	\node[part] (#1) at (#2) {};
}


\newcommand{\drawLocalPart}[2]{
\node[part,fill=blue] (#1) at ($(center)+(#2)$) {};
}

\newcommand\dissimilarity{D}
\newcommand\match{m}
\newcommand\matches{\mathcal{L}}
\newcommand\matchesCurrent{\matches_\iteration}
\newcommand\matchesPrevious{\matches_{\iteration-1}}
\newcommand\matchesAdaptive{\matchesCurrent^\ast}
\newcommand\matchesInlier{\matchesCurrent^+}
\newcommand\matchesOutlier{\matchesCurrent^-}
\newcommand\transform{H}
%\newcommand\transformInv{\transform^{-1}}
\newcommand\pointInitial{\point^0}
\newcommand\pointCurrent{\point^\iteration}
\newcommand\points{P}
\newcommand\pointsInitial{\points_0}
\newcommand\pointsCurrent{\points_\iteration}
\newcommand\numPoints{n}

%% better: (general command to vertically center horizontal material)
\newcommand*{\vcenteredhbox}[1]{\begingroup
\setbox0=\hbox{#1}\parbox{\wd0}{\box0}\endgroup}

\newcommand{\atan}{\textnormal{atan2}}
\newcommand{\median}{\textnormal{med}}

\newcommand{\iteration}{t}
\newcommand{\numParts}{N}
\newcommand{\modelPart}{p}
\newcommand{\model}{P}

\newcommand{\TP}{TP}
\newcommand{\FN}{FN}

\newcommand\interestPoint{c}
\newcommand\candidateKeypoints{C}

\newcommand{\numEstimates}{M}
\newcommand{\estimate}{\hat \modelPart}
\newcommand{\estimates}{\hat \model}
\newcommand{\estimatesTracked}{\hat \model^T}
\newcommand{\estimatesInitial}{D_0}
\newcommand{\estimatesFinal}{\hat \model^F}

\newcommand{\numInliers}{M^+}

\newcommand{\point}{x}
\newcommand{\scale}{s}
\newcommand{\scaleGT}{s_{GT}}
\newcommand{\rotationMatrix}{R}
\newcommand{\rotation}{\alpha}

\newcommand{\one}{a}
\newcommand{\two}{b}

\newcommand{\substitute}{q}
\newcommand{\residual}{\eta}

\newcommand{\noise}{\eta}

\newcommand{\modelIndex}{\pi}

\newcommand{\constellation}{H}

\newcommand{\consensus}{\omega}
\newcommand\cluster{C}
\newcommand{\consensusSet}{\cluster}
\newcommand{\valid}{\phi}
\newcommand{\bbox}{b}

\newcommand{\minimumSize}{\theta}

\newcommand{\overlap}{\phi}
\newcommand\modelDescriptors{F}
\newcommand\descriptor{f}

\newcommand\numCandidates{{\numKeypoints_\candidateKeypoints}}

\newcommand\matchThreshold{\theta}
\newcommand\matchRatio{\gamma}

\newcommand{\numSequences}{20\xspace}
\newcommand{\numSequencesOverall}{60\xspace}

\newcommand{\backgroundDescriptors}{\model^-}
\newcommand{\consensusCluster}{\votes^c}
\newcommand{\dimensionality}{d}
\newcommand{\distance}{d}
\newcommand{\distribution}{D}
\newcommand{\descriptorModel}{f}
\newcommand{\vote}{h}
\newcommand{\keypoint}{k}
\newcommand{\image}{I}
\newcommand{\keypoints}{K}
\newcommand{\keypointsCombined}{K'}
\newcommand{\modelKeypoint}{g}
\newcommand\numKeypoints{N}
\newcommand\numKeypointsModel{\numKeypoints_\model}
\newcommand\numKeypointsCombined{\numKeypoints_{\keypointsCombined}}
\newcommand\numKeypointsIteration{\numKeypoints_{\keypoints_\iteration}}
\newcommand{\numClusters}{m}
\newcommand{\keypointsMatched}{M}
\newcommand{\numImages}{n}
\newcommand{\objCenter}{\mu}
\newcommand{\keypointsTracked}{T}
\newcommand{\votes}{V}
\newcommand{\cutoffThr}{\delta}
\newcommand{\minKeypoints}{\theta}

\newcommand{\scaleHistGroup}[3]{
        \nextgroupplot
        \addplot+[black,draw=none,ybar,ybar interval,fill=black,mark=none] table {data/hist_#1.txt};
		\addplot+[blue,mark=none] plot coordinates {(0,0)} ;
		\addplot+[green,dashed,mark=none] plot coordinates {(0,0)} ;
		\draw[blue] ({axis cs:#2,0}|-{rel axis cs:0,0}) -- ({axis cs:#2,0}|-{rel axis cs:0,1});
		\draw[green,dashed] ({axis cs:#3,0}|-{rel axis cs:0,0}) -- ({axis cs:#3,0}|-{rel axis cs:0,1});
}

\newcommand{\scaleHist}[3]{
\begin{tikzpicture}
	\begin{axis}[width=3cm,%yticklabels={,,},
legend columns=-1,
legend entries={$S$,median,GT},
legend to name=mylegend,
tick label style={font=\scriptsize},
label style={font=\scriptsize},
y tick label style={/pgf/number format/.cd,sci,precision=0}
]
		\addplot+[black,draw=none,ybar,ybar interval,fill=black,mark=none] table {data/hist_#1.txt};
		\addplot+[blue,mark=none] plot coordinates {(0,0)} ;
		\addplot+[green,dashed,mark=none] plot coordinates {(0,0)} ;
		\draw[blue] ({axis cs:#2,0}|-{rel axis cs:0,0}) -- ({axis cs:#2,0}|-{rel axis cs:0,1});
		\draw[green,dashed] ({axis cs:#3,0}|-{rel axis cs:0,0}) -- ({axis cs:#3,0}|-{rel axis cs:0,1});
%		\addlegendentry{HIST}%
%		\addlegendentry{median}%
%		\addlegendentry{GT}%
	\end{axis}
\end{tikzpicture}
}

\tikzstyle{successplot}=[mark=none, ultra thick]

\newcommand{\successplot}[5]{%
\begin{tikzpicture}
  \begin{axis}[ymax=1,enlargelimits=false,font=\scriptsize,cycle multi list={solid,dashed\nextlist teal,orange,red!70!white,lime!80!black,blue!60!white,gray,pink}, width=.5\linewidth,
    title=#1,
    xlabel=#2,
    ylabel=#3]
    \input{#4}
    \input{#5}
  \end{axis}
\end{tikzpicture}%
}

%This is just for qualitative output
\newlength{\quallength}
\setlength{\quallength}{5.5cm}
\newcommand{\imagepathZ}[0]{figures/qualresults/}
\newcommand{\qualOutputZ}[4]{
\begin{tikzpicture}%
  \begin{axis}[clip marker paths=true, enlargelimits=false, ticks=none, axis equal image, width=\quallength]%
    \addplot graphics[xmin=0,ymin=0,xmax=#2,ymax=#3] {\imagepathZ/#1/input_#4.png};%
    \addplot [mark=*,mark size=1,only marks, fill=white] table [x=x, y expr=#3-\thisrow{y}] {\imagepathZ/#1/keypoints_#4.csv};%
    \addplot [mark=*,mark size=1,only marks, fill=red] table [x=x, y expr=#3-\thisrow{y}] {\imagepathZ/#1/outliers_#4.csv};%
    \addplot [thick, mark=none, blue] table [x=x, y expr=#3-\thisrow{y}] {\imagepathZ/#1/bbox_#4.csv};%
  \end{axis}%
\end{tikzpicture}%
}
\newcommand{\qualOutputBatchZ}[7]{%
  \qualOutputZ{#1}{#2}{#3}{#4}%
  \qualOutputZ{#1}{#2}{#3}{#5}%
  \qualOutputZ{#1}{#2}{#3}{#6}%
  \qualOutputZ{#1}{#2}{#3}{#7}%
}
